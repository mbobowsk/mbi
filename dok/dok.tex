\documentclass[a4paper,12pt,oneside,notitlepage,onecolumn]{article}

\usepackage{ucs}
\usepackage[utf8x]{inputenc}

\usepackage{fontenc}
\usepackage{graphicx}

\usepackage[OT4]{fontenc}
\usepackage[polish]{babel}
\usepackage{polski}
\usepackage{indentfirst}
\usepackage{graphics}

\usepackage[dvips]{hyperref}

\author{Michał Bobowski, Andrzej Dudziec}
\date{2013-05-15}
\title{Badanie podobieństwa trzech sekwencji}

\begin{document}
  \maketitle

\section{Opis algorytmu}

Algorytm realizuje badanie podobieństwa trzech sekwencji przy stałym zużyciu pamięci.
Jest uogólnioneniem algorytmu Needlmana-Wunscha, korzystającym z zasady dziel i zwyciężaj.
Podobieństwo jest określane jedynie dla całych sekwencji, a wynikiem działania algorytmu jest ciąg współrzędnych, tworzących ścieżkę w macierzy przejść (od pierwszego do ostatniego punktu).

Przeprowadzenie pełnych obliczeń zgodnie z algorytmem Needlmana-Wunscha wymaga alokacji trójwymiarowej macierzy przejść, której wymiary odpowiadają długościom badanych sekwencji nukleotydów. 
Uniemożliwia to zbadanie podobieństwa zestawu sekwencji o długości kilkuset nukleotydów.

Niniejszy algorytm redukuje ilość potrzebnej pamięci, zwiększając przy tym złożoność obliczeniową. 
Macierz przejść jest obliczana „w locie” przy użyciu dwóch aktualnych plastrów. 
Celem obliczenia macierzy jest znalezienie komórki ze środkowego plastra przez którą przechodzi najlepsze rozwiązanie, a następnie rekurencyjne wywołanie funkcji dla podzbiorów sekwencji wyznaczonych przez tą komórkę.
Warunkiem zatrzymania rekurencji jest uzyskanie przynajmniej jednej sekwencji o długości równej 1 - od tego momentu obliczana jest pełna tablica przejść.
Tym samym złożoność pamięciowa algorytmu jest ograniczona do poziomu algorytmu obliczającego dwie sekwencje, gdyż w końcowych wywołaniach istnieje potrzeba zbudowania dwuwymiarowej macierzy.

\section{Opis aplikacji}
Aplikacja została zrealizowana w języku Java w oparciu o wzorzec projektowy MVC. Zawiera następujące klasy:
\begin{itemize}
 \item Model – klasa zawierająca zmienne wpływające na działanie algorytmu, czyli macierz kar i nagród oraz wysokość kar za przerwę i wiszący nukleotyd.
 \item View – klasa odpowiedzialna za widok, interakcję z użytkownikiem i wyświetlanie wyników końcowych.
 \item Controller – klasa implementująca algorytm i przeprowadzająca obliczenia z nim związane.
 \item Cell – klasa reprezentująca pojedynczą komórkę macierzy.
 \item Tuple – krotka opakowujące współrzędne w macierzy.
\end{itemize}

Interfejs użytkownika składa się z czterech przycisków:
\begin{itemize}
 \item Matrix - pozwala na edycję macierzy kar i nagród.
 \item Open - wczytuje plik z danymi wejściowymi.
 \item Save - zapisuje wynik do pliku tekstowego.
 \item Run - uruchamia algorytm
\end{itemize}
Dodatkowo wyświetlane są sekwencje wejściowe, dopasowanie końcowe, czas obliczeń oraz maksymalne zużycie pamięci.
Formatem wejściowym dla programu są pliki tekstowe, zawierające w trzech pierwszych liniach kolejne sekwencje znaków A, C, G oraz T.

\section{Testy wydajnościowe}
Testy wydajnościowe zostały przeprowadzone na komputerze wyposażonym w procesor Intel Dual Core z taktowaniem 2,6 GHz.
Polegały one na porównaniu trzech identycznych sekwencji o założonej długości (od 100 do 1000 nukleotydów).
Dane testowe są zapisane w plikach test100.txt, ...., test1000.txt.

Warto zaznaczyć, że pomiar alokacji pamięci jest obarczony pewnym błędem.
Po pierwsze część pamięci jest pochłaniana przez elementy niezwiązane z rzeczywistymi obliczeniami (np. interfejs użytkownika), a
po drugie wirtualna maszyna Javy może nie raportować dokładnych wartości.

\begin{table}[t]
\caption{Testy wydajnościowe}
\begin{tabular}{|c|c|c|}
  \hline
  Długość sekwencji & Alokacja pamięci (MB) & Czas obliczeń (s)\\
  \hline
  100 & 86 & 0.461\\
  \hline
  200 & 184 & 1.262 \\
  \hline
  300 & 184 & 4.393 \\
  \hline
  500 & 208 & 22.849 \\
  \hline
  1000 & 493 & 349.171 \\
  \hline
\end{tabular} 
\end{table}

Wyniki testów wskazują, że dwukrotne zwiększenie długości sekwencji powoduje ok. czterokrotny wzrost czasu obliczeń.
Poziom alokacji pamięci rośnie znacznie wolniej i nie powinien być ograniczeniem nawet dla bardzo długich sekwencji.


\section{Pozostałe testy}

\end{document}
